\documentclass[12pt]{article}
\usepackage{times}
\usepackage[margin=1in]{geometry}
\usepackage{setspace}
\usepackage{cite}
\usepackage{graphicx}
\usepackage{wrapfig}
\usepackage{animate}


\newcommand{\rimage}[2]{
\begin{wrapfigure}{c}{0.5\textwidth}
    \begin{center}
\includegraphics[width=8cm]{#1}
\caption{#2}
    \end{center}
\end{wrapfigure}
}

\title{Differences in Grey Wolf Populations in 5 States}
\date{April, 2019}
\author{Harrison Brewton}

\doublespacing
\begin{document}
\maketitle
\textbf{Abstract}
This paper will analyze grey wolf populations in 
    Washington, Idaho, Minnesota, Wisconsin, and Michigan.
Differences betweens these states' policies and environment
    have led to different carrying capacities in each state
    and other summarative data in these states.
These differences will analyzed and used to suggest better policies to induce wolf populations.

\section{Grey Wolves Background}

\subsection{Habitat and Threats}
Wolves originally spanned most of the United States \cite{Biologue}.
Even during the 1800s their habitat spread all the way through Southern Michigan \cite{2015PlanMI}.
However, up until the mid-sixties the grey wolf has been the target of excessive harvesting.
This was out of farmer concerns for protecting their livestock and over sport hunting \cite{Biologue}.
Additionally, many of the traditions sources of food for wolves were diminished such as the America Bison \cite{WikiWolf}.
However, beginning in the seventies the wolf was recognized as endangered species in the lower 48 states.
The Enangered Species Act of 1973 saw some of the first early federal protection for wolf populations.
We will see how many states had near zero wolf populations during this time.
In part because of threats from humans,
wolves in the united states mostly inhabit forests, deserts, mountains, and wetlands --
areas traditionally unihabitable by humans \cite{Stats}.
Still thos common current threat to the Grey Wolf population is illegal hunting \cite{2018ReportOR}.

\subsection{Population Measurements}
Accross the five states in this survey there are a large range of different ways that population estimates are found.
Common to all states are the us of various radio collars.
These collars are used to track packs in movement and observe populations in packs through the seasosns \cite{2015ReportID}.
There are differences in the kinds of radio collars used from state to state.
While most use new GPS collars \cite{WvsInWI} \cite{2018ReportOR} \cite{2015PlanMI},
Washington still uses traditional VHF collars \cite{2019ReportWA} because of their inexpense.
Usually wolves are anesthetized after trapping at which times collars are applied.
While capturing and radioclaaring are widely used, they are not exclusively used.
In addition Idaho has used the aid of local hunters to collect data on wolf populations,
wherein they synthesize data collected from hunters with the the radio data \cite{2015ReportID}.
Minnesota has taken this strategy to extreme \cite{2018ReportMN}
Minnesota sends a survey to various groups including military bases, tribal groups, and local police departments.
These groups, distributed through the state, then collect reports on wolf sightings in a very uniform and consistent way, 
in addition to traditional radio collaring it seems this has had much success.
Some less traditional methods have been used for direct observation in Idaho, Minnesota, and Washington \cite{2015ReportID}:
aeriel investigation and placed cameras.
These, particularly the former, seem to be less successful because of the dense wilderness habitat wolves live in.
In addition to observing wolves directly a few indirect methods are used.
Wisconsin notes the use of Winter track surveys \cite{WvsInWI}.
Idaho uses thes in addition to DNA sampling of scat \cite{2015ReportID}.
Perhaps the most interesting method used in the states surveyed are howl surveys done in Wisconsin \cite{WvsInWI},
wherein estimates are produced in part by listening for the howls of wolves.

There does not seem to be one best method used among all the states.
They all do use radio collaring which allows for observation of packs,
but they also use many auxillary methods which help them gather data on the wolves in packs untracked by collared wolves.

\section{Wolf Population Dynamics of Five of the Lower-48 States}

In the following five subsections we will get a better understanding of the causes of the populations of wolves. 
For each state there will be a quick overview of that state's history, but crucially an ellaboration 
of that state's policies and their impact on their wolf populations.
In the next section we will gather what were the main problems faced by states,
    and why the population dynamics very from state to state.
To aid in this section below each of the plots of data, 
    there is a caption discussing the calculated carrying capacity.
We can also calculate the expected density of these populations through allometry.
The wolves in the surveyed states wiegh about 77 pounds for males \cite{1999ManageWI}.
Estimation of population based on allometry is difficult to get percise \cite{1995Allometric}.
Additionally, carnivores such as wolves often have lower actual densisities \cite{1995Allometric} likely because of their place in the food chain.
Using the figures in \cite{1995Allometric}, we estimate that the population density for wolves is about $0.5*10^{-2} \frac{1}{km^2}$,
    carnivorous mammals of similar weight range between $0.2*10^{-2} \frac{1}{km^2}$ and $1.1*10^{-2} \frac{1}{km^2}$
    according to that study.
While it would be convienient to be able to express through calculation from a regression, 
    the noted departure of carnivores from the regression makes regressions less useful for wolves,
    so we are limited just to this range.

\subsection{Washington}
\rimage{washington.png}{These are minimum wolf densisities, the Washington DFW suggests multiplying these by $1.12$ to adjust to real populations
    Based on this data, we see a minimum carrying capacity for Wolves sitting at about $6 \times 10^{-4}$ wolves per Square Kilometer.
}

There is some debate to degree which Washington had a substantial wolf population before Euro-American settlement \cite{WPWA}.
However, records from early Fur Trappers indicate that there probably was at least as many as there currently are.
These records report a pretty steady population despite trapping \cite{WPWA}.
Issues began in 1850 \cite{NHWA} when a bounty was set for the killing of Wolves \cite{WPWA}.
By 1930 wolves had be eradicated entirely from Washington \cite{NHWA}.
Though wolves were labeled as endangered in all 48 contiguous states, 
there was no plan drafted for te repopulation of Washington unlike states we will later look at \cite{2018VideoWA}.
Unlike other states Washington for a while did not have a long boarder with wolf dense areas  
    (most wolves in BC live North of the American boarder) \cite{NHWA},
    nor did they import wolves from these areas \cite{2018VideoWA},
    which made reintroduction difficult \cite{WPWA}. 
Howerver, there began a few wolf spottings on the Eastern side of British Columbia and from Idaho \cite{WPWA}.
And, in 2008  Washington saw its first reproducing Wolf Pack \cite{NHWA}.
With a large degree of support from the public, 75\% \cite{WPWA},
Washington began drafiting its firstplan for Wolf protection a in 2011 \cite{WPWA},
with the primary goal of reaching about 15 breading pairs.
Through the protection of livestock through reimbursment programs,
and public education about the importance of wolves to the ecosystem,
Washington has been able to grow its Wolf population at an astounding 28\% a year.
While there was some concern about the impact this growth might have on the ungulate population,
this does not seem to be a serious issue yet.
Recently they have reached this level,
but they have plateaued in  populaiton density. 
The Washington DFW explains this plaeteau as a result of all the wolf territories being taken in Eastern Washington,
and parts of the Northern cascades being difficult to repopulate with wolves \cite{2018VideoWA}.

\subsection{Idaho}
\rimage{idaho.png}{Based on the data collected, we see a carrying capacity for Idaho sitting at about $3 \times 10^{-3}$ }

Idaho gained statehood in 1890, and soon after settlers were seeking to remove the wolf population from the state to protect their livestock.
By 1915 they had pushed congress to enact a \$125,000 bill to remove all wolves from the area \cite{IdahoBackground}. 
After completely extripating wolves from the area, wovles were added to enangered species list for the Western Distict of which Idaho is a part.
Though  aslow start, the Idaho Department of Fish and Game, in coallition with the Nez Perce Tribal government, and the USFWS,
    began pushing for a program to reintroduce wolves into central Idaho and near Yellowstone National Park.
The two chief concerns for Idahoans, were the additional pressures an apex carnivore would put both on the already existent bears and mountain lions,
    and the pressure that these carnivores would put on Idaho livestock.
After an extensive environmental impact survey \cite{IdahoBackground}, 
    a plan was created to reintroduce 30 wolves from Alberta into the state \cite{WolvesReintroducedBCID}.
After this initial effort the Wolf population quickly rebounded in the state.
Reaching sustainable populations in the two target areas.
In order to allow farmers to defend their livestock,
    the State of Idaho and the Nez PErce tribe pushed for the delisting of wolves within the state \cite{2002ReportID}.
This discussion reached a federal court case in which it was decided that though the wolf population
    had reached stable populations in some areas, this was not sufficient for delisting as there remained areas unpopulated \cite{IdCase}.
The debate continued, and efforts made by the Nez Perce \cite{NezPerceLetter}
    and the state of Idaho \cite{IdahoBackground},
    wolves were successfully delisted in Idaho.
The first hunting seasosn occured in 2009 in Idaho,
    and in 2011 wolf management was movement under solely state and tribal control \cite{IdahoBackground}.
Since, the wolf population has remained at a steady density.

\subsection{Minnesota}
\rimage{minnesota.png}{Based on the data collected we see the Minnesota wolf population sitting at around $10^{-2}$}
Compared to other states in this survey, Minnesota has had a real wolf population continuosuly. 
Though, this there was a period when it came close. 
Wolves had been culturally valuable to indigenous peoples in Minnesota,
    but had also been used for furs and occacionally kileld \cite{2001ManageMN}
Though they did not present a serious concern to the first fur tappers that cam to the region,
    later settlers -- especially livestock farmers saw them as a threat to their living \cite{2001ManageMN}.
Because of this a bounty was placed on wolves in 1849.
This bounty continued up until 1965, when the first Minnesota state wolf prottections were made  \cite{2001ManageMN}.
By this point the Minnesota wolf population had fallen as low as 350.
Following both state and federall protections placed on the species in the early seventies,
    the Minnesota wolf population had rebounded to 1000 animals by 1975.
Unlike the following two states, 
Minnesota has a long boarder with the wilderness of Western Ontario.
This allowed immigration of wolf populations, sustaining the population during this period:
hence the large initial population \cite{2001ManageMN}.
The wolf population began growing at about $5\%$ per year,
this later leveled off to $3\%$ per year in 1980s.
In 2001, after concern expressed by livestock farmers restrictions on wolf populations were slightly changed.
Particularly those farmers in the Northern, more wolf populated areas of Minnesota were given the authority to kill nusance wolves near their farms,
and those in South of Minnesota were given permission to harras wolves on their farms and euthanize those that were direct threats \cite{2001ManageMN}.
This itself did not particularly harm the population of wolves (as seen in the graph).
In 2012 Minnesota held its first wolf huntin seasons. 
This combined with a decrease in prey during the period led to the sharp drop in wolf populations during this time.
Following a federal court case which saw the relisting of wolves as enangered,
thus ending the hunt allowing for a rebound to carrying capacity. 
Recently, a call has been made to revaluate the 2001 report.
Minnesota, having a steady population of wolves, seems to have steady population of wolves,
seems to have preports a lot less often \footnote{Hence the sameness of references in this subsection}.
This new report, which is to come late later in 2020, will detail particularly how Minnesota will approach future harvests.
While,
there is some push back,
the Minnesota DNR (backed by the population density graph) has made the argument that selective harvests are wise \cite{2019StarTribune}.
Such harvests both reduce pressure on local farmers, and bring in revenue from these pricey wolf-hunting licenses.

\subsection{Wisconsin}
\rimage{wisconsin.png}{Based on the data colleccted we see the Wisconsin wolf population sitting with a carrying caspacity of $5 \times 10^{-3}$}
In 1832, wolves existed throughout Wisconsin \cite{1999ManageWI}.
However, as settlement continued Wovles became a threat to the new state's livestock farmers.
In 1865, in response to this a bounty was laid on wolves,
    by 1880 wolves had been exteriminated from Southern Wisconsin,
    by 1958 wolves had been completely eliminated \cite{1999ManageWI}.
Wisconsin declared wolves to extripated species in 1975;
    however, in 1978 a pack of wolves slowly traveled from Minnesota into Wisconsin.
The population began to grow up to 27 in the early eighties.
After an early epilupic of parvovirus in 1982,
    which destroyed Wisconsin's young pup population \cite{2011WolvesWI},
    the population grew back to 34 in 1989.
Seeking to get wolves off the endangered species list in Wisconsin,
    the Wisconsin DNR issued a new plan to help the wolf population \cite{2011WolvesWI}.
The plan brought legal protection for wolf habitat and body,
    and it brought legal compensation to farmers who were affected by wolf depredation \cite{2011WolvesWI}.
This plan declared that wolves would be statwide delisted after a sustained population of 300 wolves.
After a mange-caused dip in th mid 90's Wisconsin's population rebound upto the aforementioned standard.
This caused the Wisconsin DNR to reconsider its wolf management plan.
A survey was done that found that while 78\% of hunters supported protecting wolves,
    support for growing the wolf population was dwindling.
This was seen in the fact in the less than half of farmers who felt favorable to wolves \cite{1999ManageWI}.
A management plan was drafted that while it would remain illegal to hunt wolves,
    and the DNR would continue to protect the homes of wolf populations,
    the state would also euthenize problem wolves who have been suspected of depredation \cite{1999ManageWI}.
Additionally, support would be given to farmers who had been victimized by these attacks \cite{1999ManageWI}.
By the early 2010's, support for a Wolf population had continued to fall:
    only one third of farmers supported it, 
    and many deer hunters were consider about the effect of the population on deer populations \cite{2014OpinionWI}.
This support along with other factors allowed for the passage of Act 164 in Wisocnsin which allowed for the legal hunting and traping
of wolves in particular zones. 
In 2014, upwards of 257 wolves had been harvested,
    representing a sharp decrease in wolf population for Wisconsin \cite{Milwauke}.
Following this devestating season,
    and a federal court case \cite{HuntingStoppedWI},
    Wolves were promptly reentered on the endangered list.
All hunting, trapping, and lethal defense of livestock was again outlawed.
The wolf population has since reovered and regrown in size.

\subsection{Michigan}
\rimage{michigan.png}{Based on the data colleccted we see the Wisconsin wolf population sitting with a carrying caspacity of $2 \times 10^{-3}$}

Wolves were once present throughout all of Michigan.
However, various landowners, beginning in 1840 began to push back the wolf population \cite{HistoryMI}.
By 1935, such programs brought about the entire disappearence of wolves from the Lower Peninsula.
By 1960, even the remaining wolves in the Upper Peninsula  had become rare.
In response to this Michigan passed its first state protection of wolves in 1965.
This act was ramified by the 1973 Endangered Species Act.
By the mid eighties, Michigan saw its first documented pups,
and by 1992 Michigan's grey wolf population had rebounded up to 21 \cite{HistoryMI}.
In 1990 a poll was done to gage public support for wolf recovery;
it found roughly $60\%$ of the general population supported a wolf recovery program, 
and $76\%$ of deer hunters specifically supported the prgoram \cite{1997PlanMI}.
After the aforementioned data in 1992,
public support rocketed up to only $7\%$ being against.
In the mid ninties there was a stagnet year caused by (??), which can be seen in the graph.
This, and the overwhelming public support,
spurred the Michigan DNR to ask what exactly does a sustainable wolf population look,
and how can the DNR get there.
In 1997 Michigan DNR released a plan for recovery \cite{1997PlanMI}.
This plan noted that the main challenge facing Wolf populations was the lack of support from livestock farmers \cite{1997PlanMI}.
Farmers, concerned for the protection of their animals, were prone to cull wolves that moved into their area.
In 1990, only $37\%$ of livestock farmers in the Upper Peninsula supported the recovery programs.
Indeed, the economic effect of wolves preying on livestock was the largest concern among Michigan residents according to the 1993 poll 
    \cite{1997PlanMI}.
In order to address this concern the plan pushed for large subsidy programs both to pay back farmers lost property,
    but also to support local police forces in non-mortal ways to fend off wolves.
The plan also established a minimal population of $200$ wolves over five years to serve as a benchmark for wolf population sustainability.
After reaching this goal in 2003,
a revised was implemented for the managment as opposed to recover of populations.
This new plan specifies that the increase in wolf populations is no longer the goal.
Instead, they seek to maintain a survivable wolf population in Michigan along side humans,
    and treat problem wolves on an individual level.
As seen in the graph, the density of wolves in Michigan reached a steady point at around 2 per 1000 square kilometers.

\section{What does a sustainable wolf population look like?}

Based on the data collected, 
and the background research done on these wolf populations, 
we can start to come up with answer to the question of what a sustainable wolf population looks like. 
The answer: it depends.

Let's first look a central question of what the carrying capacity should be for the species. 
First, we can conclude that all populations surveyed have reached their carrying capacity by looking at the below 
chart of year over growth rate:

\includegraphics{lambda.png}

Of the five populations analyzed we get the following data: 

\begin{center}
    \begin{tabular}{| c | c |}
        \hline 
        Statename & Carrying Capacity Density \\ 
        \hline 
        Washington & $6 \times 10^{-4} \frac{1}{km^2}$ \\ 
        Idaho & $3 \times 10^{-3} \frac{1}{km^2}$ \\ 
        Minnesota & $1 \times 10^{-2} \frac{1}{km^2}$ \\ 
        Wisconsin & $5 \times 10^{-3} \frac{1}{km^2}$ \\ 
        Michigan & $2 \times 10^{-3} \frac{1}{km^2}$ \\ 
        \hline
        Allometric & $0.5*10^{-2} \frac{1}{km^2}$ \\
        \hline 
    \end{tabular}
\end{center} 

These data have a mean of $4 \times 10^{-3}$ and a standard deviation of $32 \times 10^{-4}$.
One important thing to not when consider this average, 
especially when compared the allometric rate is the impact of human population on the data.
It is clear that the measured densities in the five states is often less than the allometric predicted densisities.
While wolf populations are known to be generalist species \cite{WPWA},
it is difficult for wolves to occupy every part of the state.
Particularly areas with large grass lands such as Washington or higher urban development such as Washington 
are notable in their lower densities, but are explainable because of these states particular characteristics. 
The number one thing that was realized while analyzing the reports from these few states was the importance of what the public wanted 
to do in the definition of sustainable.
If we look at states such as Wisconsin and Minnesota which have in the bast had Wolf seasons,
we see that the wolf population is able to pretty well rebound.
This largely means that if the right actions are taken then a state can achieve its desired population ratio.
We'll quickly consider several apsects that states seemed to have regularly considered in the five states surveyed.

\begin{enumerate}
\item [\textbf{Trapping}] 
In each of the states, but particularly Washington and Michigan,
there is some evidence of historical trapping for furs.
This normally occured in the early 1800s before heavy livestock settlement.
In the reports given, trapping of Wolves for their furs did not seem to have serious impact on wolf populations.
Now, it should be noted that times have since greatly changed, 
and the ways in which trapping might occur are different now. 
However, as some states have moved towards allowing some harvesting of wolves, 
trapping seems to be able to fit within that framework.
It seems that a sustainable wolf populaiton is one such that it can support some demand for furs.

\item [\textbf{Disease}]
One issue to consider,
especially early on in bringing a species into an area is that of disease. 
In Wisconsin, particularly, we saw two events that could have brought about the end of the wolf populations in that state. 
There was a parvovirus outbreak in the mid-eighties and a mange outbreak about ten years later.
Similar shocks can harm a wolf populaiton in the beginning of its exitence in a state.
It seems that a sustainable wolf populaiton is one such that such diseases are large enough and distributed enough such that 
disease can not decimate an entire populaiton.

\item [\textbf{Ungulates}]
A common concern raised in many states surveyed (Idaho, Minnesota, Wisconsin)
was the impact that wolves might have on the current ungulate population.
Wolves will commonly feed on these large mammals \cite{WikiWolf}.
Deer hunters, for example in Wisconsin \cite{2014OpinionWI},
    are concerned about the impact that an increase in the Wolf population would have on dears.
While there is some logic to this,
    it does not seem that currently prosperosous species such as White-tales are at grave risk from this increase in the wolf population \cite{2018VideoWA}.
There is a concern, 
however, 
about the impact a large wolf populaiton might have on endangered ungulates.
We can see such a scenario occur to a wide degree currently in British Columbia.
In order to recover Caribou, an endangered species in the area, a wolf cull that would target upward of $80\%$ of the current wolf populaiton \cite{BCWolfCull}.
While this is something to keep in mind,
the current Wolf populations are stable and do not seem to create serious problems for other endangered species.
It seems that wolves normally do create unsustainable populations over animals, 
but that a sustainable wolf population should be monitored for this posibility. 

\item [\textbf{Harvesting}]
We saw wolf harvests in three of states through this survey: Idaho, Minnesota, and Wisconsin. 
In reading the background for each of these states,
it is clear that there was some substantial concenr about the impact that these hunts would have on wolf populaitons.
And while some hunst such as the one in Wisconsin had to be cut short \cite{Milwauke},
it seems that limited permits and zoning of where hunting could occur allowed for Wolf populaitons to rebound afterwards.
The conclusion of each of these was that a harvest for a few consecutive years was acceptable, but that it might not be sustainble to conintuously do.
Between pushes for harvests and court cases against them,
it seems that wolf populaitons do fluculate, 
but do generally rebound to a consistent rate.
So it does seem that a harvest may not be able to happen every year, at least a consistent rate. 
So, a sustainable wolf populaiton is one that supports some degree of a demand for harvest,
but can also rebound afterwards.

\item [\textbf{Livestock}]
The largest concern expressed in most states is the impact of wolf populations on livestock.
Since the beginning (and of often before) of each of the five states observed,
Farmers in the area saw wolves (often rightly so) as a direct threat to their livestock through depredation.
In order to resolve this issue they implemented severe bounties on wolves, 
which ended up decimating and in all but Minnesota extripating wolves from the area.
Because there is some deman for wolves to exist within a state, 
such complete elimination obviously falls out an appropriate defintion of sustainability.
However, wolves also must be kept at a level such that they do not threaten farmers livestock to ean extent that 
reimburssments (as used by all states) become infeasible.
Additionally, many states have argued that instead of eliminating all wolves only nusance packs should be euthanized 
\cite{2015PlanMI} \cite{2014Depredation} \cite{2018ReportMN}.
\end{enumerate}

One thing that is clear from this research that what makes a sustainble wolf population 
is really dependent on what the values are of the citizens of a state.
An argument could be made that the wolf populations of the early 1900s were sustainable.
Though wolves were entirely extripated from four of the five states,
this was sustainable to the demands of the time.
However, now that wolves have value to both unters and tourists,
it is important to follow Leopolds lead and look to what people actually want to judge sustainability.

One question we might to answer is what do different harvesting patterns look like in a wolf population.
If we look at the few states that have had harvests they have done so for a few years and then not so for a few years.
Often this is because of legal battles between states and federal definitions of endangered,
but it does beg an interesting question:
what is the sustainability of periodic harvests.
We can describe a periodic harvest with a 3-tuple: $(h, p, r)$.
Here, 
$h$ is the number of years that a harvest occurs for, 
$p$ is the number of years that no harvest occurs, 
$r$ is the percent of expected wolves to be taken by the harvest.
And we can describe a logistic process:

$$ P_{t+1} = gP_t (1 - P_t/k)$$

Where $k$ is the carrying capacity, and $g$ is growth rate on off years
and the growth rate minus the cull rate on harvest years.
In order to investigate this model I created a program to investigate the space of possible harvests

\section{Conclusion}

\bibliography{main}{}
\bibliographystyle{plain}
\end{document}